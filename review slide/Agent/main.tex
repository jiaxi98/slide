\documentclass[aspectratio=169]{beamer}
\usetheme{boxes}
\usepackage{essay-def}
\usepackage{bm}
\usepackage{booktabs}
\usepackage{amsfonts}
\usepackage{amssymb}
\usepackage{amsmath}
\usepackage{amsthm}
\usepackage{comment}
\usepackage{enumitem}
\usepackage{geometry}
\usepackage{algorithmic}
\usepackage{algpseudocode}
\usepackage{algorithmicx}
\usepackage{graphicx}
\usepackage{subcaption}
\usepackage{tikz}
\usepackage{physics}
\usepackage{xcolor}
\usepackage{enumitem}
\setitemize{label=$\circ$}
\setbeamertemplate{itemize subitem}{\textbullet}
\usepackage[version=4]{mhchem}
\geometry{left=1cm,right=1cm}
\title[Code Agent]{21 Days with Claude Code}
\author[J. Zhao]{Jiaxi Zhao (NUS)}
\date[\today]{\today}

% TODO: add reference to each slide, better use a concise notion
\begin{document}
% \par \setlength{\parindent}{2em}

\begin{frame}
\titlepage
\end{frame}


\begin{frame}{How does AI agent use tool?}
	\begin{itemize}
		\item MCP. 
	\end{itemize}
\end{frame}


\begin{frame}{Agent literature}
	\begin{itemize}
		\item 1. 
	\end{itemize}
\end{frame}


\begin{frame}{Reverse engineering the Claude Code}
	\begin{itemize}
		\item System prompts
		\item Agentic workflow 
	\end{itemize}
\end{frame}


\begin{frame}{Good features}
	\begin{itemize}
		\item 1. TODO list
		% \begin{figure}[ht] 
		% 	\centering 
		% 	\includegraphics[width=.3\textwidth]{fig/qrcode.jpg} 
		% \end{figure}
	\end{itemize}
\end{frame}


\begin{frame}{I have tested Claude on...}
	\begin{itemize}
		\item Scientific computing code reading (99\%)
		\item Modify docker file and environment (99\%)
		\item Write unit test given clear interface and complete instructions
		\item Debug my hand-written spectral solver for 2D NS equation (40\%)
		\item Set hydra for multiple run experiments (20\%)
		\item Write Poisson solver over radial grid (NA)
	\end{itemize}
\end{frame}


\begin{frame}{Claude Code apts to ...}
	\begin{itemize}
		\item 1. Writing unit test given clear interface and complete instructions.
		% \begin{figure}[ht] 
		% 	\centering 
		% 	\includegraphics[width=.3\textwidth]{fig/qrcode.jpg} 
		% \end{figure}
	\end{itemize}
\end{frame}


\begin{frame}{Claude Code as a tool}
	\begin{itemize}
		\item Many people may ignore or even realize: Claude Code highly
		decrease the barrier for most of the tasks:
		\begin{itemize}
			\item Learn a heavy repo
			\item Handle the environment setup
		\end{itemize}
	\end{itemize}
\end{frame}




% \begin{frame}{Potential improvement}
% 	\begin{itemize}
% 		\item Using greedy algorithm to optimize for more Gaussian modes, i.e.
% 		one can use the optimization results for less modes as the warm-start
% 		for more modes
% 		\item The approximation interval $[0, M]$ can be done adaptively according
% 		to the relative position of two Gaussian modes
% 		\item Motivated by the Boys function, can we design other decomposition
% 		of the Coulomb kernel?
% 		\begin{equation*}
% 			\frac{1}{r} = C\int_{\mbR} e^{-r^2t^2} dt
% 		\end{equation*}
% 	\end{itemize}
% \end{frame}


% \begin{frame} % Use [allowframebreaks] to allow automatic splitting across slides if the content is too long
%     \frametitle{References}
 
%     \begin{thebibliography}{99} % Beamer does not support BibTeX so references must be inserted manually as below, you may need to use multiple columns and/or reduce the font size further if you have many references
%         \footnotesize % Reduce the font size in the bibliography
 
% 		\bibitem[migbs]{migbs}
% 		Gill, Peter MW.
% 		\newblock Molecular integrals over gaussian basis functions.
% 		\newblock \emph{Advances in quantum chemistry. Vol. 25. Academic Press,
% 		1994. 141-205.}

% 		\bibitem[3dgs]{3dgs}
% 		Kerbl, Bernhard, et al
% 		\newblock 3D Gaussian Splatting for Real-Time Radiance Field Rendering
% 		\newblock \emph{ACM Trans. Graph. 42.4 (2023): 139-1.}

% 		\bibitem[S.A. and Q.L., 2024]{nigbms}
% 		S. Arisaka and Q. Li (2024)
% 		\newblock Accelerating Legacy Numerical Solvers by Non-intrusive Gradient-based Meta-solving
% 		\newblock \emph{International Conference on Machine Learning 2024}
 
        
%     \end{thebibliography}
% \end{frame}

\end{document}