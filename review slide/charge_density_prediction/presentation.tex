\documentclass[aspectratio=169]{beamer}
\usetheme{boxes}
\usepackage{amsmath}
\usepackage{amssymb}
\usepackage{graphicx}
\usepackage{subcaption}
\usepackage{tikz}
\usepackage{physics}
\usepackage{xcolor}
\usepackage{booktabs}

\title{Recent Advances in Charge Density Prediction}
\author{Jiaxi Zhao (NUS)}
\date{\today \\
Multiscale Modelling and Machine Learning Seminar}

\begin{document}

\documentclass[paper slide]{beamer}
\usetheme{Boadilla}
\usepackage{essay-def}
\usepackage{bm}
\usepackage{amsfonts}
\usepackage{amssymb}
\usepackage{amsmath}
\usepackage{amsthm}
\usepackage{comment}
\usepackage{subcaption}
\usepackage{geometry}
\usepackage{algorithmic}
\usepackage{algpseudocode}
\usepackage{algorithmicx}
\geometry{left=1cm,right=1cm}
    \title[Interview]{Interview for MatMat group}
\author[J. Zhao]{Jiaxi Zhao}
\date[\today]{}
\begin{document}
\par \setlength{\parindent}{2em}

\begin{frame}
\titlepage
\end{frame}


\begin{frame}{}
	\noindent
	\small
	Many machine-learning assisted scientific computing methods have an
	iterative nature, e.g., turbulence modeling, XC functional:
  \begin{equation*}
         \partial_t\mfu = \mcL(\mfu, \mfy, t), \quad \mfy = \phi(\mfu, t),
				 \quad \min_{\theta} \sum_{n} \norml \phi_{\theta}(\mfu^{(n)}) - \mfy^{(n)} \normr^2,
	\end{equation*}
	The stability and a-posteriori performance are not satisfactory.
	\begin{itemize}
		\item 1. Tangent-space regularization method (\textit{published in SISC})
		\begin{itemize}
			\item · First mathematical formulation and analysis.
			\item · Non-intrusive and differentiable regularization.
			\item · Significant improvement over dynamics-agnostic methods.
			\item · Deployment to practical urban environment simulation (ongoing).
		\end{itemize}
		\item 2. Generative subgrid-scale model (\textit{accepted by ICLR 2025 MLMP})
		\item 3. Numerical anlysis of the turbulence modeling (multiscale,
		data-imbalance, multivaluedness)
		\item 4. Distorted plane-wave via normalizing flow (adaptive basis set, smaller
		cutoff energy, comparable accuracy)
		\item 5. Adaptive Gaussian basis set (efficient electron integral, differentiable basis sets)
	\end{itemize}
\end{frame}
 
 
\begin{frame}{Mathematics, physics, and programming background}
	\noindent
  Mathematics \& Physics:
	\begin{itemize}
    \item 1. Mathematics, graduate level course on computational math and analysis: 
		numerical analysis, (numerical) PDE, numerical linear algebra, stochastic analysis, etc.
		\item 2. Physics (quantum mechanics, quantum chemistry, solid-state physics, fluid dynamics).
  \end{itemize}
	\noindent
	Programming:
	\begin{itemize}
    \item 1. Core contributor of the \textit{Jrystal} package (pseudopotential,
		accuracy test modules, exploring the machine-learning XC functionals).
    \item 2. Extensive experience with deep-learning framework and models (PyTorch, JAX, generative models,
		differentiable programming).
    \item 3. Familiar with various open-source packages (PySCF, OpenFOAM).
  \end{itemize}
\end{frame}

 
\begin{frame}{Research vision and contribution to MatMat group}
	\noindent
	Research vision:
	\begin{itemize}
		\item Machine-learning tools can advance scientific computing (differentiable programming,
		data-centric viewpoint, generative modeling)
		\item Numerical analysis for hybrid algorithms (stability analysis, error estimation, 
		uncertainty quantification)
	\end{itemize}
	I aim to contribute to the MatMat group research in the following two aspects:
	
	1. Gradient-accelerated inverse materials design 
		\begin{itemize}
			\item · Combining stabilization algorithm with inverse design.
			\item · Generative model guided by differentiable DFT calculation.
		\end{itemize}
	
	2. Estimation of simulation errors
		\begin{itemize}
			\item · Numerical analysis related to pseudo-potential and XC functional.
			\item · Machine-learning XC functionals and its related problems.
		\end{itemize}
\end{frame}


% \begin{frame} % Use [allowframebreaks] to allow automatic splitting across slides if the content is too long
%     \frametitle{References}
 
%     \begin{thebibliography}{99} % Beamer does not support BibTeX so references must be inserted manually as below, you may need to use multiple columns and/or reduce the font size further if you have many references
%         \footnotesize % Reduce the font size in the bibliography
 
% 		\bibitem[BDI]{bdi}
% 		M. Benjamin, S. Domino, and G. Iaccarino
% 		\newblock Neural Networks for Large Eddy Simulations of Wall-bounded Turbulence: Numerical Experiments and Challenges
% 		\newblock \emph{The European Physical Journal E}

% 		\bibitem[Zhao 2024]{ds}
%         J. Zhao and Q. Li (2024)
%         \newblock Mitigating Distribution Shift in Machine Learning-augmented Hybrid Simulation
%         \newblock \emph{Arxiv preprint https://arxiv.org/pdf/2401.09259}

%         \bibitem[S.A. and Q.L., 2024]{nigbms}
%         S. Arisaka and Q. Li (2024)
%         \newblock Accelerating Legacy Numerical Solvers by Non-intrusive Gradient-based Meta-solving
%         \newblock \emph{International Conference on Machine Learning 2024}
 
        
%     \end{thebibliography}
% \end{frame}

\end{document}
\section{E3NN}

\begin{frame}{E3NN: Density fitting}
  \begin{itemize}
    \item A consistent basis for all the charge density, not the case for DFT.
    \item {\color{red} \textbf{Density fitting}}: fit the density to a predefined basis set, def2-universal-JFIT\footnotemark.
  \end{itemize}
  \begin{equation*}
    \rho(\mathbf{r}) = \sum_b C_b \phi_b^{\text{basis}}(\mathbf{r}),
  \end{equation*}

  The objective function is to minimize:
    $\mathcal{L} = \frac{1}{N} \sum_i \left(C_b - \widehat{C_b}\right)^2$.
  \begin{itemize}
    \item 3 hidden layers.
    \item Input reps = 2x0e (2 channels l=0) with hydrogen ([1, 0])
    and oxygen ([0, 1]).
    \item Hidden reps = 125x0o + 125x0e + 40x1o + 40x1e + 25x2o + 25x2e + 15x3o + 15x3e.
    \item Output reps = 7x0e + 4x1e + 2x2e + 1x3e.
  \end{itemize}

  \footnotetext[6]{
    Rackers, Joshua A., et al. "Cracking the quantum scaling limit with machine
    learned electron densities." arXiv preprint arXiv:2201.03726 (2022).
  }
\end{frame}


\begin{frame}{The effect of cluster size on density prediction}
  This is one of the first lines of work:
  \begin{itemize}
    \item Does not train on large dataset for density prediction.
    \item Training on specific systems to push the frontier of quantum chemistry
    calculations.
  \end{itemize}
  The y-axis is the relative $L^1$-error.
  \begin{figure}
    \includegraphics[width=0.6\textwidth]{figures/e3nn_2.jpg}
  \end{figure}
\end{frame}
\section{SCDP: Spherical Channel Density Prediction}

\begin{frame}{SCDP Architecture Overview}
    \begin{itemize}
        \item Basis-based approach with following ingredients:
        \begin{itemize}
            \item Virtual nodes for non-local electronic structures
            \item Even-tempered Gaussian basis sets
            \item High-capacity equivariant spherical channel network (eSCN).
        \end{itemize}
        \item Charge density representation with trainable scale parameters:
        \begin{align*}
            \rho(\mathbf{r}) &= \sum_a \sum_l^{N_a} \sum_{m = -l}^{l} c_{alm}
            \Phi_{\alpha,l,m,\mathbf{r}_a}(\mathbf{r}, s_{al}) \\
            \Phi_{\alpha,l,m,\mathbf{r}_i}(\mathbf{r}, s) &= 
            z_{\alpha,l,s} \exp(-s \cdot \alpha |\mathbf{r} - \mathbf{r}_a|^2) 
            |\mathbf{r} - \mathbf{r}_a|^l Y_{lm}(
                \widehat{\mathbf{r} - \mathbf{r}_a})
        \end{align*}
        \item Model prediction, scaling factors $s_{al} \in (0.5, 2.0)$:
        \begin{align*}
            \{c_{alm}, s_{al}\} &= F(\{(\mathbf{r}_a, Z_a)\}).
        \end{align*}
    \end{itemize}
\end{frame}


\begin{frame}{Virtual Nodes and Basis Sets}
    \begin{itemize}
        \item Virtual nodes:
        \begin{itemize}
            \item Placed at bond midpoints.
            \item Use oxygen basis functions
            \item SE(3)-equivariant placement
        \end{itemize}
        \item Even-tempered Gaussian basis for better accuracy:
        \begin{align*}
            \alpha_k &= \alpha \cdot \beta^k \text{ for } k = 0,1,2,...,N_l
        \end{align*}
        \item  Reducing SO(3) convolution to SO(2): $O(L^6) \rightarrow O(L^3)$.
        \item  The basis set is not orthonormal, the coefficients depend on the
        cutoff radius.
        \item Two-stage training for stability caused by the scale factors
        on the exponent:
        \begin{itemize}
            \item pre-train the model with fixed basis set exponents
            \item fine-tune the prediction model with a small learning rate with
            the learning for scaling factors enabled.
        \end{itemize}
    \end{itemize}
\end{frame}


\begin{frame}{Model Architecture}
    \begin{itemize}
        \item Backbone: eSCN (equivariant spherical channel network)
        \begin{itemize}
            \item $\{x_a\} = \text{eSCN}(\{(\mathbf{r}_a, Z_a)\})$
            \item Complexity: $O(L^3)$ vs $O(L^6)$ for tensor products
            \item Features: Multi-channel spherical harmonics
            \item Example: $128\times0e + 128\times1o + 128\times2e + 128\times3o$
        \end{itemize}
        \item Prediction layers:
        \begin{align*}
            \{c_{alm}, h_i\} &= \text{FullyConnectedTensorProduct}(x_i, x_l) \\
            s_{al} &= \frac{C_1}{1 + \exp(-\text{Linear}(h_i) + \ln C_2)} + C_3
            \in [C_1, C_3].
        \end{align*}
        \item Training objective:
        \begin{align*}
            \mathcal{L} &= \mathbb{E}_{\mathbf{r}\in\text{Data}}[|\rho(\mathbf{r}) - \hat{\rho}(\mathbf{r})|]
        \end{align*}
    \end{itemize}
\end{frame}


% \begin{frame}{pseudocode}
%     \begin{figure}
%         \includegraphics[width=\textwidth]{figures/scdp_3.jpg}
%     \end{figure}
% \end{frame}


\begin{frame}{Performance Analysis}
    \begin{itemize}
        \item NMAE: 0.178\% on QM9 test set
        \item 31.7x faster than ChargE3Net
        \item Flexible accuracy-efficiency trade-off
    \end{itemize}
    \begin{figure}
        \begin{subfigure}{0.48\textwidth}
        \includegraphics[width=\textwidth]{figures/scdp_2.jpg}
    \end{subfigure}
    \begin{subfigure}{0.48\textwidth}
        \includegraphics[width=\textwidth]{figures/scdp_1.jpg}
    \end{subfigure}
    \end{figure}
\end{frame}

\section{ChargE3Net}

\begin{frame}{ChargE3Net Architecture Overview}
    \begin{itemize}
        \item E(3)-equivariant graph neural network
        \item Key components:
        \begin{itemize}
            \item Graph construction with atoms and probe points
            \item Higher-order equivariant features (up to L=4)
            \item Message passing between atoms and probes
            \item Equivariant convolution operations
        \end{itemize}
        \item Input: Atomic species and positions
        \item Output: Charge density at probe points
    \end{itemize}
\end{frame}

\begin{frame}{Graph Construction}
    \begin{itemize}
        \item Vertices:
        \begin{itemize}
            \item Atoms: One-hot encoding of atomic number
            \item Probe points: Initialized as zero scalar
        \end{itemize}
        \item Edges:
        \begin{itemize}
            \item Atom-atom: Unidirectional, cutoff 4Å
            \item Atom-probe: Directed from atoms to probes
        \end{itemize}
        \item Periodic boundary conditions supported
    \end{itemize}
\end{frame}

\begin{frame}{Message Passing Architecture}
    \begin{itemize}
        \item Two types of convolutions:
        \begin{itemize}
            \item Conv$_{\text{atom}}$: Bidirectional between atoms
            \item Conv$_{\text{probe}}$: From atoms to probes
        \end{itemize}
        \item Each layer:
        \begin{itemize}
            \item Updates atom representations
            \item Updates probe representations
            \item Uses tensor product operations
        \end{itemize}
        \item Final layer: Regression to predict charge density
    \end{itemize}
\end{frame}

\begin{frame}{Performance on Benchmark Datasets}
    \begin{itemize}
        \item Materials Project (MP):
        \begin{itemize}
            \item ChargE3Net: 0.523\% ϵmae
            \item DeepDFT: 0.799\% ϵmae
            \item invDeepDFT: 0.859\% ϵmae
        \end{itemize}
        \item QM9:
        \begin{itemize}
            \item ChargE3Net: 0.206\% ϵmae
            \item OrbNet-Equi: 0.284\% ϵmae
            \item DeepDFT: 0.357\% ϵmae
        \end{itemize}
        \item NMC:
        \begin{itemize}
            \item ChargE3Net: 0.061\% ϵmae
            \item DeepDFT: 0.060\% ϵmae
        \end{itemize}
    \end{itemize}
\end{frame}

\begin{frame}{Impact on DFT Calculations}
    \begin{itemize}
        \item SCF step reduction:
        \begin{itemize}
            \item MP materials: 26.7\% reduction
            \item GNoME materials: 28.6\% reduction
        \end{itemize}
        \item Non-self-consistent property prediction:
        \begin{itemize}
            \item 40\% of materials: energy errors < 1 meV/atom
            \item 70\% of materials: forces < 0.03 eV/Å
            \item 76\% of materials: band gaps within chemical accuracy
        \end{itemize}
        \item Linear scaling O(N) with system size
    \end{itemize}
\end{frame}

\begin{frame}{Higher-Order Features Analysis}
    \begin{itemize}
        \item Performance vs. rotation order:
        \begin{itemize}
            \item L=0: Basic scalar features
            \item L=1: Vector features
            \item L=2,3,4: Higher-order tensor features
        \end{itemize}
        \item Channel distribution:
        \begin{itemize}
            \item N$_{\text{channels}} = \lfloor 500/(L+1) \rfloor$
            \item Equal representation size across orders
        \end{itemize}
        \item Consistent improvement with increasing L
    \end{itemize}
\end{frame}

\begin{frame}{Angular Variance Analysis}
    \begin{itemize}
        \item High angular variance materials:
        \begin{itemize}
            \item Example: Cs(H2PO4)
            \item Strong covalent bonding
            \item Significant L=4 improvement
        \end{itemize}
        \item Low angular variance materials:
        \begin{itemize}
            \item Example: Rb2Sn6
            \item Primarily ionic interactions
            \item Similar L=0 and L=4 performance
        \end{itemize}
        \item Metric $\zeta$ for angular variance:
        \[
        \zeta(G) = 1 - \frac{\sum_{\vec{g}_k \in G} |\nabla\rho(\vec{g}_k) \cdot \hat{r}_{ki}|}{\sum_{\vec{g}_k \in G} ||\nabla\rho(\vec{g}_k)||}
        \]
    \end{itemize}
\end{frame} 
% \section{Uni-3DAR: Unified 3D Generation and Understanding}

\begin{frame}{Uni-3DAR Overview}
    \begin{itemize}
        \item Tokenization-based approach
        \item Key components:
        \begin{itemize}
            \item Hierarchical octree compression
            \item Fine-grained structural tokenization
            \item Masked next-token prediction
        \end{itemize}
        \item Unifies:
        \begin{itemize}
            \item 3D structure generation
            \item Property prediction
            \item Multi-modal tasks
        \end{itemize}
    \end{itemize}
\end{frame}

\begin{frame}{Hierarchical Tokenization}
    \begin{itemize}
        \item Octree-based compression:
        \begin{itemize}
            \item Coarse-to-fine subdivision
            \item Non-empty cell detection
            \item Level-wise tokenization
        \end{itemize}
        \item Fine-grained tokenization:
        \begin{itemize}
            \item Atom types and positions
            \item In-cell coordinate discretization
            \item Structural details
        \end{itemize}
        \item 2-level subtree compression:
        \begin{itemize}
            \item 8 subcells → 1 token
            \item 256 possible states
            \item 8x reduction in tokens
        \end{itemize}
    \end{itemize}
\end{frame}

\begin{frame}{Masked Next-Token Prediction}
    \begin{itemize}
        \item Challenge: Dynamic token positions
        \item Solution:
        \begin{itemize}
            \item Token duplication
            \item Masked token replacement
            \item Position-aware prediction
        \end{itemize}
        \item Benefits:
        \begin{itemize}
            \item Handles varying token positions
            \item Maintains causal sampling
            \item Improves prediction accuracy
        \end{itemize}
    \end{itemize}
\end{frame}

\begin{frame}{Unified Framework}
    \begin{itemize}
        \item Single-frame generation:
        \begin{itemize}
            \item Unconditional generation
            \item Property-conditioned generation
            \item Text-guided generation
        \end{itemize}
        \item Multi-frame generation:
        \begin{itemize}
            \item Molecular dynamics
            \item Pocket-based generation
            \item Frame-by-frame prediction
        \end{itemize}
        \item Understanding tasks:
        \begin{itemize}
            \item Token-level properties
            \item Structure-level properties
            \item Cross-modal tasks
        \end{itemize}
    \end{itemize}
\end{frame}

\begin{frame}{Performance Analysis}
    \begin{itemize}
        \item Generation tasks:
        \begin{itemize}
            \item Up to 256\% relative improvement
            \item 21.8x faster inference
            \item Better quality and diversity
        \end{itemize}
        \item Understanding tasks:
        \begin{itemize}
            \item Competitive with specialized models
            \item Effective transfer learning
            \item Multi-task learning benefits
        \end{itemize}
    \end{itemize}
\end{frame}

\begin{frame}{Efficiency Optimizations}
    \begin{itemize}
        \item Training:
        \begin{itemize}
            \item FlashAttention with bfloat16
            \item Sequence packing
            \item Efficient memory usage
        \end{itemize}
        \item Inference:
        \begin{itemize}
            \item KV-cache acceleration
            \item Paired token generation
            \item GPU utilization optimization
        \end{itemize}
    \end{itemize}
\end{frame}

\begin{frame}{Cross-Modal Applications}
    \begin{itemize}
        \item Protein folding:
        \begin{itemize}
            \item Sequence to structure
            \item Multi-frame generation
            \item Property prediction
        \end{itemize}
        \item Crystal structure prediction:
        \begin{itemize}
            \item PXRD-guided generation
            \item NMR signal conditioning
            \item Property optimization
        \end{itemize}
    \end{itemize}
\end{frame} 

      
\begin{frame}{Discussions}
    \begin{itemize}
        \item Incorperate more symmetries: space-group symmetry, point-group symmetry of crystal.
        \item Comparison with approaches of canonical transformations. This type of method is
        is restricted to be probe-based.
        \item Different approaches to equivariance in robot community: equivariant vision.
        \item {\color{red} \textbf{Sobolev training for downstream tasks where the gradient information of the density
        is required.}}
        \item Does it worth to pursue the equivariance for the model? Seems not necessary as
        the AlphaFold evolves.
    \end{itemize}
\end{frame}

\end{document} 