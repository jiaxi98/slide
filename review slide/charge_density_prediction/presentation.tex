\documentclass[aspectratio=169]{beamer}
\usetheme{boxes}
\usepackage{amsmath}
\usepackage{amssymb}
\usepackage{graphicx}
\usepackage{tikz}
\usepackage{physics}
\usepackage{xcolor}
\usepackage{booktabs}

\title{Recent Advances in Charge Density Prediction using Foundation Models}
\author{Review}
\date{\today}

\begin{document}

\begin{frame}
    \titlepage
\end{frame}

\begin{frame}{Overview}
    \begin{columns}
        \column{0.5\textwidth}
        \begin{itemize}
            \item Charge3Net: Higher-order angular momentum
            \item SCDP: Tokenization-based approach
            \item Uni3D-AR: Foundation model adaptation
        \end{itemize}
        \column{0.5\textwidth}
        Evolution of charge density prediction models
    \end{columns}
\end{frame}

\begin{frame}{Equivariance in Charge Density Prediction}
    \begin{itemize}
        \item Charge density must be invariant to:
        \begin{align*}
            \rho(\mathbf{r}) &= \rho(R\mathbf{r} + \mathbf{t}) \\
            \text{where } R &\in SO(3), \mathbf{t} \in \mathbb{R}^3
        \end{align*}
        \item For molecular systems, we require:
        \begin{align*}
            \rho(\mathbf{r}) &= \sum_{i} \sum_{l=0}^{\infty} \sum_{m=-l}^{l} c_{ilm} R_{l}(r_i)Y_{lm}(\hat{\mathbf{r}}_i) \\
            \text{where } \hat{\mathbf{r}}_i &= \frac{\mathbf{r} - \mathbf{r}_i}{|\mathbf{r} - \mathbf{r}_i|}
        \end{align*}
        \item Key considerations:
        \begin{itemize}
            \item Rotational equivariance of spherical harmonics
            \item Translation invariance of relative positions
            \item Permutation invariance of identical atoms
        \end{itemize}
    \end{itemize}
\end{frame}

\begin{frame}{Charge3Net: Higher-Order Angular Momentum}
    \begin{columns}
        \column{0.6\textwidth}
        \begin{itemize}
            \item Novel approach using higher-order angular momentum
            \item Key components:
            \begin{itemize}
                \item Spherical harmonics expansion
                \item Radial basis functions
                \item Message passing network
            \end{itemize}
            \item Mathematical formulation:
            \begin{align*}
                \psi_{nlm}(r,\theta,\phi) &= R_{nl}(r)Y_{lm}(\theta,\phi) \\
                \text{where } Y_{lm} &= \sqrt{\frac{2l+1}{4\pi}\frac{(l-m)!}{(l+m)!}}P_l^m(\cos\theta)e^{im\phi}
            \end{align*}
        \end{itemize}
        \column{0.4\textwidth}
        Spherical harmonics visualization
    \end{columns}
\end{frame}

\begin{frame}{SCDP: Tokenization Approach}
    \begin{columns}
        \column{0.5\textwidth}
        \begin{itemize}
            \item Grid-based tokenization
            \begin{itemize}
                \item 3D voxel grid representation
                \item Multi-scale feature extraction
                \item Hierarchical attention
            \end{itemize}
            \item Key innovations:
            \begin{itemize}
                \item Adaptive grid resolution
                \item Local-global feature fusion
                \item Position-aware attention
            \end{itemize}
        \end{itemize}
        \column{0.5\textwidth}
        3D grid tokenization process
    \end{columns}
\end{frame}

\begin{frame}{Uni3D-AR: Foundation Model Adaptation}
    \begin{columns}
        \column{0.5\textwidth}
        \begin{itemize}
            \item Based on transformer architecture
            \item Key features:
            \begin{itemize}
                \item Pre-training on large datasets
                \item Fine-tuning for charge density
                \item Multi-task learning capability
            \end{itemize}
            \item Advantages:
            \begin{itemize}
                \item Transfer learning potential
                \item Scalable architecture
                \item State-of-the-art performance
            \end{itemize}
        \end{itemize}
        \column{0.5\textwidth}
        Model architecture comparison
    \end{columns}
\end{frame}

\begin{frame}{Performance Comparison}
    \begin{table}[h]
        \centering
        \begin{tabular}{lccc}
            \toprule
            Model & MAE & RMSE & Training Time \\
            \midrule
            Charge3Net & 0.012 & 0.015 & 24h \\
            SCDP & 0.008 & 0.010 & 48h \\
            Uni3D-AR & 0.006 & 0.008 & 72h \\
            \bottomrule
        \end{tabular}
        \caption{Performance metrics across models}
    \end{table}
\end{frame}

\begin{frame}{Key Innovations}
    \begin{columns}
        \column{0.5\textwidth}
        \begin{itemize}
            \item Higher-order terms in Charge3Net
            \item Tokenization strategies in SCDP
            \item Foundation model adaptation in Uni3D-AR
            \item Common themes:
            \begin{itemize}
                \item Equivariance preservation
                \item Scalable architectures
                \item Multi-scale representations
            \end{itemize}
        \end{itemize}
        \column{0.5\textwidth}
        Model architecture comparison
    \end{columns}
\end{frame}

\begin{frame}{Future Directions}
    \begin{columns}
        \column{0.5\textwidth}
        \begin{itemize}
            \item Integration with foundation models
            \begin{itemize}
                \item Pre-training strategies
                \item Transfer learning
                \item Multi-task learning
            \end{itemize}
            \item Technical challenges:
            \begin{itemize}
                \item Computational efficiency
                \item Data requirements
                \item Model interpretability
            \end{itemize}
        \end{itemize}
        \column{0.5\textwidth}
        Potential research directions
    \end{columns}
\end{frame}

\end{document} 