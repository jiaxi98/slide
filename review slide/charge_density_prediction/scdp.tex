\section{SCDP: Spherical Channel Density Prediction}

\begin{frame}{SCDP Architecture Overview}
    \begin{itemize}
        \item Orbital-based approach with three key ingredients:
        \begin{itemize}
            \item Virtual nodes for non-local electronic structures
            \item Even-tempered Gaussian basis sets
            \item High-capacity equivariant neural networks (eSCN)
        \end{itemize}
        \item Input: Atomic species and positions
        \item Output: Charge density through orbital coefficients
    \end{itemize}
\end{frame}

\begin{frame}{Mathematical Formulation}
    \begin{itemize}
        \item Charge density representation:
        \begin{align*}
            \rho(\mathbf{r}) &= \sum_i^N \sum_j^{N_b^i} \sum_{m = -l_{ij}}^{l_{ij}} c_{ijm}^i \Phi_{\alpha_{ij},l_{ij},m,\mathbf{r}_i}(\mathbf{r}, s_{ij}) \\
            \text{where } \Phi_{\alpha,l,m,\mathbf{r}_i}(\mathbf{r}, s) &= z_{\alpha,l,s} \exp(-s \cdot \alpha r^2) r^l Y_{lm}(\hat{\mathbf{r} - \mathbf{r}_i})
        \end{align*}
        \item Model prediction:
        \begin{align*}
            \{c_{ijm}, s_{ij}\} &= F(A, R) \\
            \text{where } A &= \{a_i | i = 1,...,N\} \text{ (atomic types)} \\
            R &= \{\mathbf{r}_i | i = 1,...,N\} \text{ (coordinates)}
        \end{align*}
    \end{itemize}
\end{frame}

\begin{frame}{Equivariance Guarantees}
    \begin{itemize}
        \item SE(3) equivariance preservation:
        \begin{itemize}
            \item Input coordinates: $R \in SO(3), \mathbf{t} \in \mathbb{R}^3$
            \item Virtual node placement: Bond midpoints (SE(3)-equivariant)
            \item Basis coefficients: SE(3)-equivariant
            \item Scaling factors: SE(3)-invariant
        \end{itemize}
        \item Architecture components:
        \begin{itemize}
            \item eSCN backbone: SO(3) convolutions reduced to SO(2)
            \item Tensor product layers: Preserve equivariance
            \item Point-wise spherical non-linearity
        \end{itemize}
    \end{itemize}
\end{frame}

\begin{frame}{Model Architecture}
    \begin{itemize}
        \item Backbone: eSCN (equivariant spherical channel network)
        \begin{itemize}
            \item Complexity: $O(L^3)$ vs $O(L^6)$ for tensor products
            \item Features: Multi-channel spherical harmonics
            \item Example: $128\times0e + 128\times1o + 128\times2e + 128\times3o$
        \end{itemize}
        \item Prediction layers:
        \begin{align*}
            \{c_{ijm}, h_i\} &= \text{FullyConnectedTensorProduct}(x_i, x_i) \\
            s_{ij} &= \frac{C_1}{1 + \exp(-\text{Linear}(h_i) + \ln C_2)} + C_3
        \end{align*}
        \item Training objective:
        \begin{align*}
            \mathcal{L} &= \mathbb{E}_{\mathbf{r}\in\text{Data}}[|\rho(\mathbf{r}) - \hat{\rho}(\mathbf{r})|]
        \end{align*}
    \end{itemize}
\end{frame}

\begin{frame}{Fully Connected Tensor Product Operation}
    \begin{itemize}
        \item Mathematical definition:
        \begin{align*}
            (U^{(\ell_1,p_1)} \otimes V^{(\ell_2,p_2)})^{(\ell_o,p_o)}_{cm_o} &= \sum_{m_1=-\ell_1}^{\ell_1} \sum_{m_2=-\ell_2}^{\ell_2} C^{(\ell_o,m_o)}_{(\ell_1,m_1)(\ell_2,m_2)} U^{(\ell_1,p_1)}_{cm_1} V^{(\ell_2,p_2)}_{cm_2}
        \end{align*}
        where:
        \begin{itemize}
            \item $|\ell_1 - \ell_2| \leq \ell_o \leq |\ell_1 + \ell_2|$
            \item $p_o = p_1p_2$
            \item $C^{(\ell_o,m_o)}_{(\ell_1,m_1)(\ell_2,m_2)}$ are Clebsch-Gordan coefficients
        \end{itemize}
        \item Properties:
        \begin{itemize}
            \item Preserves SO(3) equivariance
            \item Combines features of different angular momenta
            \item Output channels determined by tensor product rules
        \end{itemize}
        \item Implementation:
        \begin{itemize}
            \item Pre-computed Clebsch-Gordan coefficients
            \item Efficient sparse tensor operations
            \item Channel-wise feature combination
        \end{itemize}
    \end{itemize}
\end{frame}

\begin{frame}{Virtual Nodes and Basis Sets}
    \begin{itemize}
        \item Virtual nodes:
        \begin{itemize}
            \item Placed at bond midpoints
            \item Use oxygen basis functions
            \item SE(3)-equivariant placement
        \end{itemize}
        \item Even-tempered Gaussian basis:
        \begin{align*}
            \alpha_k &= \alpha \cdot \beta^k \text{ for } k = 0,1,2,...,N_l
        \end{align*}
        \item Learnable parameters:
        \begin{itemize}
            \item Basis coefficients $c_{ijm}$
            \item Scaling factors $s_{ij} \in (0.5, 2.0)$
            \item Two-stage training for stability
        \end{itemize}
    \end{itemize}
\end{frame}

\begin{frame}{Performance Analysis}
    \begin{itemize}
        \item Accuracy:
        \begin{itemize}
            \item NMAE: 0.178\% on QM9 test set
            \item 31.7x faster than ChargE3Net
            \item Flexible accuracy-efficiency trade-off
        \end{itemize}
        \item Efficiency:
        \begin{itemize}
            \item Linear scaling with system size
            \item GPU-optimized implementation
            \item Efficient basis function evaluation
        \end{itemize}
    \end{itemize}
\end{frame}

\begin{frame}{Downstream Applications}
    \begin{itemize}
        \item Property prediction:
        \begin{itemize}
            \item Total energy
            \item Forces
            \item Electronic structure
        \end{itemize}
        \item Materials discovery:
        \begin{itemize}
            \item High-throughput screening
            \item Property optimization
            \item Structure prediction
        \end{itemize}
        \item Molecular dynamics:
        \begin{itemize}
            \item Force field generation
            \item Trajectory simulation
            \item Property evolution
        \end{itemize}
    \end{itemize}
\end{frame} 