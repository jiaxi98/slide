\section{ChargE3Net}

\begin{frame}{ChargE3Net Architecture Overview}
    \begin{itemize}
        \item E(3)-equivariant graph neural network
        \item Key components:
        \begin{itemize}
            \item Graph construction with atoms and probe points
            \item Higher-order equivariant features (up to L=4)
            \item Message passing between atoms and probes
            \item Equivariant convolution operations
        \end{itemize}
        \item Input: Atomic species and positions
        \item Output: Charge density at probe points
    \end{itemize}
\end{frame}

\begin{frame}{Graph Construction}
    \begin{itemize}
        \item Vertices:
        \begin{itemize}
            \item Atoms: One-hot encoding of atomic number
            \item Probe points: Initialized as zero scalar
        \end{itemize}
        \item Edges:
        \begin{itemize}
            \item Atom-atom: Unidirectional, cutoff 4Å
            \item Atom-probe: Directed from atoms to probes
        \end{itemize}
        \item Periodic boundary conditions supported
    \end{itemize}
\end{frame}

\begin{frame}{Message Passing Architecture}
    \begin{itemize}
        \item Two types of convolutions:
        \begin{itemize}
            \item Conv$_{\text{atom}}$: Bidirectional between atoms
            \item Conv$_{\text{probe}}$: From atoms to probes
        \end{itemize}
        \item Each layer:
        \begin{itemize}
            \item Updates atom representations
            \item Updates probe representations
            \item Uses tensor product operations
        \end{itemize}
        \item Final layer: Regression to predict charge density
    \end{itemize}
\end{frame}

\begin{frame}{Performance on Benchmark Datasets}
    \begin{itemize}
        \item Materials Project (MP):
        \begin{itemize}
            \item ChargE3Net: 0.523\% ϵmae
            \item DeepDFT: 0.799\% ϵmae
            \item invDeepDFT: 0.859\% ϵmae
        \end{itemize}
        \item QM9:
        \begin{itemize}
            \item ChargE3Net: 0.206\% ϵmae
            \item OrbNet-Equi: 0.284\% ϵmae
            \item DeepDFT: 0.357\% ϵmae
        \end{itemize}
        \item NMC:
        \begin{itemize}
            \item ChargE3Net: 0.061\% ϵmae
            \item DeepDFT: 0.060\% ϵmae
        \end{itemize}
    \end{itemize}
\end{frame}

\begin{frame}{Impact on DFT Calculations}
    \begin{itemize}
        \item SCF step reduction:
        \begin{itemize}
            \item MP materials: 26.7\% reduction
            \item GNoME materials: 28.6\% reduction
        \end{itemize}
        \item Non-self-consistent property prediction:
        \begin{itemize}
            \item 40\% of materials: energy errors < 1 meV/atom
            \item 70\% of materials: forces < 0.03 eV/Å
            \item 76\% of materials: band gaps within chemical accuracy
        \end{itemize}
        \item Linear scaling O(N) with system size
    \end{itemize}
\end{frame}

\begin{frame}{Higher-Order Features Analysis}
    \begin{itemize}
        \item Performance vs. rotation order:
        \begin{itemize}
            \item L=0: Basic scalar features
            \item L=1: Vector features
            \item L=2,3,4: Higher-order tensor features
        \end{itemize}
        \item Channel distribution:
        \begin{itemize}
            \item N$_{\text{channels}} = \lfloor 500/(L+1) \rfloor$
            \item Equal representation size across orders
        \end{itemize}
        \item Consistent improvement with increasing L
    \end{itemize}
\end{frame}

\begin{frame}{Angular Variance Analysis}
    \begin{itemize}
        \item High angular variance materials:
        \begin{itemize}
            \item Example: Cs(H2PO4)
            \item Strong covalent bonding
            \item Significant L=4 improvement
        \end{itemize}
        \item Low angular variance materials:
        \begin{itemize}
            \item Example: Rb2Sn6
            \item Primarily ionic interactions
            \item Similar L=0 and L=4 performance
        \end{itemize}
        \item Metric $\zeta$ for angular variance:
        \[
        \zeta(G) = 1 - \frac{\sum_{\vec{g}_k \in G} |\nabla\rho(\vec{g}_k) \cdot \hat{r}_{ki}|}{\sum_{\vec{g}_k \in G} ||\nabla\rho(\vec{g}_k)||}
        \]
    \end{itemize}
\end{frame} 