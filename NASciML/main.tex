\documentclass{beamer}
\usetheme{Boadilla}
\usepackage{essay-def}
\usepackage{bm}
\usepackage{amsfonts}
\usepackage{amssymb}
\usepackage{amsmath}
\usepackage{amsthm}
\usepackage{comment}
\usepackage{geometry}
\geometry{left=1cm,right=1cm}
    \title[SciML]{Numerical analysis for scientific machine learning}
\author[J. Zhao]{Jiaxi Zhao}
\date{17th Apr, 2024}
\begin{document}
\par \setlength{\parindent}{2em}

\begin{frame}
\titlepage

\end{frame}

\begin{frame}{Brief overview}
	Scientific machine learning is applied to a broad range of computational problems, including
	\begin{itemize}
		\item 1. Steady-state simulations \& inverse problems
		\item 2. Unsteady (transient) simulations
		\item 3. Eigenvalue problems
	\end{itemize}
	Many classical physical processes belong to the first two categories well the third class contains mostly quantum problems.
\end{frame}

\begin{frame}{Steady-state simulations \& inverse problems}
	%#TODO: The RANS equation here does not exactly fit into the framework given here,
	% maybe it is better to use the inverse scattering problem (Calderon problem) as an example.
	Most of the steady-state simulations and inverse problems can be formulated as 
	\begin{equation*}
		\begin{aligned}
			\mcL(\mfu, \mfy) = 0, 	\quad \mfy= \phi_{\theta}(\mfu).
		\end{aligned}
	\end{equation*}
	One example of special interests is the \textbf{Reynolds-averaged Navier-Stokes equation}:
	\begin{equation*}
		\begin{aligned}
	(\la \mfU \ra \cdot \nabla)\la \mfU \ra + \frac{\p \la \mfu u_j \ra}{\p x_j} &= -\frac{1}{\rho}\nabla p + \nu \Delta \la \mfU \ra,		\\
	\nabla \cdot \la \mfU \ra &= 0.
	\end{aligned}
	\end{equation*}
	Great flexibility to choose the scheme for solving these problems: pseudo transient, Newton method,
	iterative methods, etc.
\end{frame}

\begin{frame}{Unsteady (transient) simulations}
	%#TODO: Qianxiao mentioned the structure stability here, which is different from the numerical stability of the numerical PDE we discussed. 
	% In other words, when doing the simulation the data-driven surrogate may not respect the structure of the dynamics or destroy the structure of the original dynamics
	% if the original one is marginally stable or contains metastability. I think this is a good point as a theoretical question but can hardly be a question which
	% is of practical importance. At least I do not say any important application.
	Two leading applications are \textbf{Large eddy simulation} and \textbf{Molecular dynamics}.
	\bequn\label{non-linear}
		\begin{aligned}
			\p_t \mfu = \mcL(\mfu, \mfy, t), 	\quad \mfy = \phi_{\theta}(\mfu, t).
		\end{aligned}
	\eequn
	\begin{itemize}
		\item Temporal and spatial discretization
		\item Stability and convergence analysis
	\end{itemize}
\end{frame}

\begin{frame}{Eigenvalue problem}
	\begin{itemize}
		\item \textbf{Density functional theory}
		\begin{equation*}
			E = E_{\mathrm{Kin}}[\{\psi_{i}^\sigma\}] + E_{\mathrm{Har}}[\rho] + E_{\mathrm{Ext}}[\rho] + E_{\mathrm{XC}}[\rho]
		\end{equation*}
		\item \textbf{Quantum many-body problem (Quantum Monte Carlo)}
		\begin{equation*}
			\min_{\theta} \frac{\la \psi_{\theta}(\mfr_1, \cdots, \mfr_n)|\wht H| \psi_{\theta}(\mfr_1, \cdots, \mfr_n)\ra}{\la \psi_{\theta}(\mfr_1, \cdots, \mfr_n)| \psi_{\theta}(\mfr_1, \cdots, \mfr_n)\ra}
		\end{equation*}
	\end{itemize}
	There are tons of works focusing on learning the exchange-correlation functional $E_{\mathrm{XC}}[\cdot]$ and plugging into the SCF loops to improve the accuracy, which
	can be formalized into the following eigenvalue problem:
	\begin{equation*}
		\min_{\norml \mfv \normr = 1} \mfv^T (H_0 + \phi_{\theta}) \mfv.
	\end{equation*}
\end{frame}

\begin{frame}{Key features}
	There are several key features in the numerical analysis of SciML:
	\begin{itemize}
		\item 1. Low-dimensional structures induced by the data, as for real applications the data is always sparse but structured. The data-driven surrogate can only be 
		trusted in a low-dimensional space.
		\item 2. Statistical properties of the data-driven surrogate, such as the bias and asymptotic normality. These could be integrated with the numerical analysis to 
		obtain a precise analysis of simulation dynamics.
	\end{itemize}
\end{frame}

% \begin{frame}
% 	%#TODO: add references on CFD and DFT
% 	\begin{itemize}
% 		\item Pfau, David, et al. "Ab initio solution of the many-electron Schrödinger equation with deep neural networks." Physical Review Research 2.3 (2020): 033429.
% 		\item 
% 	\end{itemize}
% \end{frame}

\end{document}