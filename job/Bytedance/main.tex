\documentclass[aspectratio=169]{beamer}
\usetheme{boxes}
\usepackage{essay-def}
\usepackage{bm}
\usepackage{booktabs}
\usepackage{amsfonts}
\usepackage{amssymb}
\usepackage{amsmath}
\usepackage{amsthm}
\usepackage{comment}
\usepackage{enumitem}
\usepackage{geometry}
\usepackage{algorithmic}
\usepackage{algpseudocode}
\usepackage{algorithmicx}
\usepackage{graphicx}
\usepackage{subcaption}
\usepackage{tikz}
\usepackage{physics}
\usepackage{xcolor}
\usepackage[version=4]{mhchem}
\geometry{left=1cm,right=1cm}
\title[ML4DFT]{Advancing Density Functional Theory Calculation with Machine Learning}
\author[J. Zhao]{Jiaxi Zhao (NUS)}
\date[\today]{\today\\
Job talk @ Bytedance}

% TODO: add reference to each slide, better use a concise notion
\begin{document}
\par \setlength{\parindent}{2em}

\begin{frame}
\titlepage
\end{frame}


\begin{frame}{Bio}
	\begin{itemize}
		\item 2016--2020: Undergraduate in Peking univerisity majored in mathematics;
		\item 2021--2026 (expected): Ph.D. in national university of Singapore majored in
		applied mathematics, focusing on 
		\item 2024--present: Research intern in Sea AI Lab to improve the jax-based
		solid-state DFT solver \textit{Jrystal} using machine-learning toolbox.
	\end{itemize}
\end{frame}


% for long talk
\begin{frame}{Many-body Schr\"odinger equation}
	After Born–Oppenheimer approximation, the many-body Schr\"odinger equation
	is given by:
	\begin{equation*}
		\begin{aligned}
			\hat{\opH^e} & \Psi(\mfr_1, \mfr_2, ..., \mfr_N) = E
			\Psi(\mfr_1, \mfr_2, ..., \mfr_N),		\\
			\hat{\opH^e} & = -\half \Delta_{\mfr} - \sum_{i=1}^N\sum_{j=1}^{n_A} \frac{Z_j}
			{\norml \mfr_i - \mfR_j \normr_2} + \frac{1}{2} \sum_{i \neq j}^N
			\frac{1}{r_{ij}}		\\
			& = \sum_{i=1}^N \wht{\operatorname{h}}(\text{i}) + \frac{1}{2} \sum_{i \neq j}^N
			\frac{1}{r_{ij}}
		\end{aligned}
	\end{equation*}
	Solving this equation, we can obtain the ground state energy, potential
	energy surface (for geometry optimization and transition states search),
	band structure for the solid state system, and etc.
\end{frame}

% for long talk
\begin{frame}{Mean-field approximation: Slater determinant}
	Pauli exclusion principle requires following antisymmetric constraint on the
	many-body wave function:
	\begin{equation*}
		\Psi_{\text{HF}}(..., \mfr_j, ..., \mfr_i, ...) = 
		- \Psi_{\text{HF}}(..., \mfr_i, ..., \mfr_j, ...)
	\end{equation*}

	The many-body wave function of the Hartree-Fock theory is given by a single
	Slater determinant:
	\begin{equation*}
		\Psi_{\text{HF}}(\mfr_1, ..., \mfr_N) = \frac{1}{\sqrt{N!}} 
		\begin{vmatrix}
			\phi_1(\mfr_1) & \phi_1(\mfr_2) & \cdots & \phi_1(\mfr_N) \\
			\phi_2(\mfr_1) & \phi_2(\mfr_2) & \cdots & \phi_2(\mfr_N) \\
			\vdots & \vdots & \ddots & \vdots \\
			\phi_N(\mfr_1) & \phi_N(\mfr_2) & \cdots & \phi_N(\mfr_N)
		\end{vmatrix}
	\end{equation*}
\end{frame}

% for long talk
\begin{frame}{Total energy}
	\begin{equation*}
		\begin{aligned} & E[\Psi^{HF}] = 
			\left\langle\Psi^{HF}|\hat{\opH^e}|\Psi^{HF}
			\right\rangle =  \sum_{i=1}^N \int\text{d}\mathbf{r}_i
		  \phi_i^*(\mathbf{r}_i)
			\wht{\operatorname{h}}(\text{i}) \phi_i(\mathbf{r}_i) \\ 
			&+ \frac{1}{2N(N - 1)} \sum_{i \neq j}^N \sum_{k \neq l}^N \iint
			\mathrm{d}\mathbf{r}_i
			\text{d}\mathbf{r}_j \phi_k^*(\mathbf{r}_i)\phi_l^*(\mathbf{r}_j)
			\frac{1}{|\mathbf{r}_i-\mathbf{r}_j|}\phi_k(\mathbf{r}_i)
			\phi_l(\mathbf{r}_j) \\ 
			&- \frac{1}{2N(N - 1)} \sum_{i \neq j}^N \sum_{k \neq l}^N \iint
			\text{d}\mathbf{r}_i
			\text{d}\mathbf{r}_j\phi_k^*(\mathbf{r}_i)\phi_l^*(\mathbf{r}_j)
			\frac{1}{|\mathbf{r}_i-\mathbf{r}_j|}\phi_k(\mathbf{r}_j)
			\phi_l(\mathbf{r}_i)  \\
			& = E(\phi_1, \phi_2, \cdots, \phi_N).
		\end{aligned}
	\end{equation*}
	Minimizing the total energy will give the ground state configuration which is
	consistent with the self-consistent field approach. The double-electron
	integrations $O(N^4)$ are the dominant part of computational cost.
\end{frame}

\begin{frame}{Mathematical formulation}
	Variational formulation of the Hartree-Fock theory:
	\begin{equation*}
		\begin{aligned}
			& E(\phi_1, \phi_2, \cdots, \phi_N) = 
	\sum_{i=1}^N \int\text{d}\mathbf{r}_i
		  \phi_i^*(\mathbf{r}_i) (-\frac{1}{2}\Delta - \sum_{\text{atom } a}
			\frac{Z_a}{|\mathbf{r}_i - \mathbf{A}_a|})(\mathbf{r}_i))
			 \phi_i(\mathbf{r}_i) \\ 
			&+ \frac{1}{2N(N - 1)} \sum_{i \neq j}^N \sum_{k \neq l}^N \iint
			\mathrm{d}\mathbf{r}_i
			\frac{1}{|\mathbf{r}_i-\mathbf{r}_j|}
			\left(\underbrace{\phi_l(\mathbf{r}_j)\phi_l^*(\mathbf{r}_j)}_{\rho_l(\mathbf{r}_j)}
			\text{d}\mathbf{r}_j \overbrace{\phi_k^*(\mathbf{r}_i)\phi_k(\mathbf{r}_i)
			}^{\rho_k(\mathbf{r}_i)} - \underbrace{\phi_k^*(\mathbf{r}_i)\phi_l^*(\mathbf{r}_j)
			\phi_k(\mathbf{r}_j)\phi_l(\mathbf{r}_i)}_{\text{Exchange term}}\right)  \\
      & \int\text{d}\mathbf{r}
		  \phi_i^*(\mathbf{r}) \phi_j(\mathbf{r}) = \delta_{ij}.
		\end{aligned}
	\end{equation*}
	\begin{itemize}
		\item 1. Mean-field approximation of the quantum many-body Schr\"odinger equation.
		\item 2. Direct optimization of the total energy; self-consistent field (SCF) method.
		\item 3. Discretization of the wave function: $\phi_j = \sum_{i\in I} c_{ji}\varphi_i$.
	\end{itemize}
\end{frame}

\begin{frame}{Basis set\footnotemark}
	\begin{itemize}
		\item Slater basis functions:
		\begin{equation*}
			\varphi_{a, A, \alpha}^{\opS} \equiv (x - A_x)^{a_x} (y - A_y)^{a_y}
			(z - A_z)^{a_z} e^{-\alpha \norml \mfr - \mfA \normr_2}.
		\end{equation*}
		\item Gaussian basis functions:
		\begin{equation*}
			\varphi_{a, A, \alpha}^{\opG} \equiv (x - A_x)^{a_x} (y - A_y)^{a_y}
			(z - A_z)^{a_z} e^{-\alpha \norml \mfr - \mfA \normr_2^2}.
		\end{equation*}
		\item Contracted Gaussian basis functions (PySCF, QChem):
		\begin{equation*}
			\varphi_{a, A, \alpha, k}^{\text{c}\opG} \equiv \sum_{k=1}^{K_A}
			D_{aAk}(x - A_x)^{a_x}(y - A_y)^{a_y}(z - A_z)^{a_z}
			e^{-\alpha_k \norml \mfr - \mfA \normr_2^2}.
		\end{equation*}
		\item Numerical atomic orbitals (Abacus):
		\begin{equation*}
			\varphi_{a, A, \alpha, k} \equiv \sum_{L}^{L_A}
			Y_L(\widehat{\mathbf{r} - \mathbf{A}})
			f_L(\norml \mathbf{r} - \mathbf{A} \normr_2^2).
		\end{equation*}
	\end{itemize}
	\footnotetext[1]{Gill, Peter MW. "Molecular integrals over gaussian basis
	functions." Advances in quantum chemistry. Vol. 25. Academic Press, 1994.
	141-205.}
\end{frame}

% for long talk
\begin{frame}{A primer on molecular integral over Gaussian basis functions}
	It suffices to evaluate the fundamental integrals:
	\begin{equation*}
		\iint e^{-\alpha \norml \mfr_1 - \mfA \normr_2^2}
		e^{-\beta \norml \mfr_2 - \mfB \normr_2^2} f(r_{12})
		e^{-\gamma \norml \mfr_1 - \mfC \normr_2^2}
		e^{-\delta \norml \mfr_2 - \mfD \normr_2^2}d\mfr_1d\mfr_2.
	\end{equation*}
	Integrations involving higher angular momentum can be calculated by
	taking the derivation w.r.t. $\mfr_1, \mfr_2$ and recursive formula.

	{\color{red}Caveat: the recursive formula becomes very complex for higher
	angular momentum and thus computationally expensive.}
\end{frame}

\begin{frame}{Adaptive Gaussian basis}
	% Classically, basis set of increasing complexity is used to 
	% \begin{itemize}
	% 	\item 1. STO-n 
	% 	\item 2. 6-311G*
	% 	\item 3. cc-pVDZ
	% \end{itemize}

	{\color{red}\textbf{Basis set optimization}} is also a hot topic for research.

	Adaptive Gaussian basis with optimizable mean and covariance:
	\begin{equation*}
		\phi_i = \sum_{j\in I} c_{ij}\mcN(\mfr; \mu_j, \Sigma_j).
	\end{equation*}
	In general, all the optimizable parameters are $\mu \in \mbR^{|I|\times 3},
	\Sigma \in \lp \mbS_+^3 \rp^{|I|}, C \in \mbC^{|I|\times N}$. $|I|$ is the
	size of the basis set, $N$ is the number of electrons.

	\begin{itemize}
		\item 1. More flexibility to model the wave functions.
		\item 2. Analytic Gaussian integral.
	\end{itemize}
	\footnotetext{Kerbl, Bernhard, et al. "3D Gaussian Splatting for Real-Time
	Radiance Field Rendering." ACM Trans. Graph. 42.4 (2023): 139-1.}
\end{frame}

\begin{frame}{Orthonormality and overlap matrix}
	The orthonormality of the orbitals in SCF is guaranteed by the eigensolver.
	In total energy minimization, we have to handle this explicitly.

	As the orbitals are supposed to be orthonormal, it can be written as
	\begin{equation*}
		\opC^{\dagger} \opS \opC = \opI.
	\end{equation*}
	where $\opS$ is the overlap matrix defined as:
	\begin{equation*}
		\operatorname{S}_{ij} = \la \mcN\lp \mfr; \mathbf{\mu_i},
    \Sigma_i \rp \rv \left. \mcN\lp \mfr;
    \mathbf{\mu_j}, \Sigma_j \rp \ra.
	\end{equation*}
	The orthonormality is enforced by QR decomposition modified by the
	Cholesky factorization of $\opS$.
\end{frame}

\begin{frame}{Coulomb and external energy}
	Why difficulites? No analytic formula for
	\begin{equation*}
		\la \mcN\lp \mfr_1 | \mathbf{\mu_1}, \Sigma_1 \rp \rv \frac{1}{
				\lv \mfr_1 - \mfr_2 \rv}\lv \mcN\lp \mfr_2 | \mathbf{\mu_2}, \Sigma_2
				\rp \ra
	\end{equation*}

	Our primary idea is to approximate the Coulomb kernel by a series of
	Gaussian modes:
	\begin{equation*}
		\frac{1}{r} = \sum_i c_i e^{-\alpha_i r^2},
	\end{equation*}
	which is done by minimizing the following
	\begin{equation*}
		\begin{aligned}
			\min_{c_i, \alpha_i} \int_{B_M} \lp \frac{1}{|\mfr|} - \sum_i c_i
			e^{-\alpha_i |\mfr|^2} \rp^2 d\mfr = 4\pi\int_0^M \lp 1 - \sum_i c_i
			re^{-\alpha_i r^2} \rp^2 dr
		\end{aligned}
	\end{equation*}
	Then we fix the number of Gaussian modes and optimize the analytic loss
	function via gradient-based optimization.
\end{frame}

% for long talk
% \begin{frame}{Analysis of the nonlinear optimization problem}
% 	\begin{equation*}
% 		\begin{aligned}
% 			& \ \int_0^M \lp 1 - \sum_i c_i re^{-\alpha_i r^2} \rp^2 dr 
% 			\overset{u = kr}{=} & \ \frac{1}{k}\int_0^{kM} \lp 1 - \sum_i \frac{c_i}{k} u 
% 			e^{-\frac{\alpha_i}{k^2} u^2} \rp^2 du.
% 		\end{aligned}
% 	\end{equation*}
% 	The optimal solution for different $M$ can be obtained by rescaling the optimal
% 	coefficient and variance of the Gaussian modes.

% 	However, optimizing such problem to a relatively low threshold, e.g. $10^{-5}$
% 	is quite difficult.
% 	\begin{figure}[h]
% 		\centering
% 		\includegraphics[width=0.5\linewidth]{fig/convergence.pdf}
% 	\end{figure}
% \end{frame}

\begin{frame}{Coulomb kernel decomposition}
	\begin{figure}[h]
		\centering
		\includegraphics[width=\linewidth]{fig/modes_M50i20.pdf}
	\end{figure}
\end{frame}

\begin{frame}{More on double electron integrals}
	\begin{equation*}
		\begin{aligned}
			& \ \la \mcN\lp \mfr_1 | \mathbf{\mu_1}, \Sigma_1 \rp \rv \frac{1}{
				\lv \mfr_1 - \mfr_2 \rv}\lv \mcN\lp \mfr_2 | \mathbf{\mu_2}, \Sigma_2
				\rp \ra     \\
			= & \ \int\int d\mfr_1d\mfr_2 \frac{1}{\sqrt{(2\pi)^6\det(\Sigma_1)
			\det(\Sigma_2)}} \sum_i c_i  \exp\lb -\half\lp (\mfr_1 - \mu_1)^T
			\Sigma_1^{-1}(\mfr_1 - \mu_1) \right.\right.		\\
				& \ \left.\left. + (\mfr_2 - \mu_2)^T\Sigma_2^{-1}(\mfr_2 - \mu_2)\rp
				- (\mfr_1 - \mfr_2)^T \alpha_i \operatorname{I} (\mfr_1 - \mfr_2) \rb
			\\
			= & \ \int\int d\mfr \frac{\sum_i c_i s(i)\exp\lb -\half(\mfr - \mu(i))^T \begin{pmatrix}
				\Sigma_1^{-1} + 2\alpha_i \operatorname{I} & -2\alpha_i \operatorname{I}
				\\
				-2\alpha_i \operatorname{I} & \Sigma_2^{-1} + 2\alpha_i \operatorname{I}
				\end{pmatrix} (\mfr - \mu(i)) \rb}{\sqrt{(2\pi)^6\det(\Sigma_1)
			\det(\Sigma_2)}}.
		\end{aligned}
	\end{equation*}
	\begin{itemize}
		\item 1. solving a linear system of size $3\times 3$, MVP of size $3\times 3$, calculating
		the determinant of a $3\times 3$ matrix.
		\item 2. The FLOPs count for a single ERI evaluation with one $\alpha$ is around 60, comparing
		to classical method: (ss|ss), (ps|ps), (pp|pp) requires 33, 58, 1326 FLOPs for a single evaluation.
	\end{itemize}
\end{frame}

\begin{frame}{Parameterization of Gaussian basis}
	We parameterize the Gaussian basis as
	\begin{equation*}
		\Sigma_1 = \opU_1 \opD_1 \opU_1^T, \quad \Sigma_2 = \opU_2 \opD_2 \opU_2^T,
	\end{equation*}
	where $\opU_i$ is obtained from the QR decomposition and the positivity of
	the diagonal element of $\opD_i$ is enforced by the softplus function.

	This is proved to be more robust than parametrization by Cholesky
	factorization.
\end{frame}

\begin{frame}{Numerical experiments}
	\begin{table}[tb]\label{table:accuracy-2d-ot}
		\label{table:density-fitting}
		\centering
		\Large
		\resizebox{0.95\columnwidth}{!}{
		\begin{tabular}{P{4cm} P{2cm} P{2cm} P{2cm} P{2cm} P{2cm} P{2cm} P{2cm}}
		\toprule [1.5pt]
		\parbox{4cm}{  } &   \parbox{2cm}{ \centering e\_kin + e\_ext} &
		\parbox{2cm}{\centering  e\_kin}
		& \parbox{2cm}{\centering  e\_ext} & 
		\parbox{2cm}{ \centering e\_nuc } & \parbox{2cm}{ \centering 
		e\_coul + e\_exc} & \parbox{2cm}{ \centering 
		e\_coul} & \parbox{2cm}{ \centering e\_exc} & \parbox{2cm}
		{\centering  e\_tot}   \\ \midrule[1.5pt]
		
		\parbox{4cm}{OUR ($\ce{H2}$), 10} & -2.5062 & 1.1254 & -3.6316 & 0.7138 & 0.6591 &
		1.3182 & -0.6591 & \textbf{-1.1334} \\ \midrule[0.5pt]
	
		% \parbox{3cm}{$n_g= 10, \epsilon=10^{-3}$} & -2.4905 & 1.1070 & -3.5975
		% & 0.7138 & 0.6549 & 1.3097 & -0.6549 & -1.1219 \\ \midrule[0.5pt]
	
		% \parbox{3cm}{$n_g = 20, \epsilon=10^{-3}$} & -2.4897 & 1.1069 & -3.5966
		% & 0.7138 & 0.6547 & 1.3094 & -0.6547 & -1.1213 \\ \midrule[0.5pt]
	
		% \parbox{3cm}{$n_g = 20, \epsilon=10^{-4}$} & -2.4917 & 1.1095 & -3.6012
		% & 0.7138 & 0.6550 & 1.3100 & -0.6550 & -1.1229 \\ \midrule[0.5pt]
	
		% \parbox{3cm}{$n_g = 10, \epsilon=10^{-4}$} & -2.4909 & 1.1078 & -3.5987
		% & 0.7138 & 0.6550 & 1.3100 & -0.6550 & -1.1221 \\ \midrule[0.5pt]
	
		\parbox{4cm}{HF (STO-3G, 2)} 
		& -2.5049 &  * & * & 0.7138 & 0.6745 & * & * & -1.1167
		\\ \midrule[0.5pt]
	
		\parbox{4cm}{HF (6-31G, 4)} 
		& -2.4902 &  * & * & 0.7138 & 0.6497 & * & * & -1.1267
		\\ \midrule[0.5pt]
	
		\parbox{4cm}{HF (6-311G, 6)} 
		& -2.4924 &  * & * & 0.7138 & 0.6507 & * & * & \textbf{-1.1280}
		\\ \midrule[0.5pt]
	
		% \parbox{3cm}{d4ft (6-31g, lda)} 
		% & -2.482366 & * & * & 0.713754 & * & 1.280327 & -0.552556 & -1.038646
		% \\ \midrule[0.5pt]
	
		% \parbox{3cm}{pyscf} 
		% & -2.480171 & * & * & 0.713754 & * & 1.275792 & -0.554381 & -1.047201
		% \\ \midrule[0.5pt] \midrule[0.5pt]
	
		% \parbox{3cm}{OUR (\ce{CH4})} & -78.6437 & 37.8278 & -116.4715 & 13.4477
		% & 25.8757 & 32.3750 & -6.4993 & -39.3203 \\ \midrule[0.5pt]
	
		% \parbox{3cm}{$n_g = 10, \epsilon=10^{-3}$} & -78.3927 & 39.6854 &
		% -118.0781 & 13.4477 & 25.6199 & 32.2481 &
		% -6.6282 & -39.3251 \\ \midrule[0.5pt]
	
		% \parbox{3cm}{$n_g = 20, \epsilon=10^{-3}$} & -79.3790 & 39.9992 &
		% -119.3782 & 13.4477 & 26.01440 & 32.5979 &
		% -6.5835 & -39.9169 \\ \midrule[0.5pt]
	
		% \parbox{3cm}{$n_g = 20, \epsilon=10^{-4}$} & -79.2670 & 39.8448 &
		% -119.1118 & 13.4477 & 25.9379& 32.5050 &
		% -6.5671 & -39.8813 \\ \midrule[0.5pt]
	
		% \parbox{3cm}{$n_g = 10, \epsilon=10^{-4}$} & -77.5650 & 39.5177 &
		% -117.0827 & 13.4477 & 24.9640 & 31.5523 &
		% -6.5883 & -39.1532 \\ \midrule[0.5pt]
	
		\parbox{4cm}{OUR ($\ce{CH4}$), 22} & -79.6713 & 40.0987 & -119.7700 & 13.4477
		& 26.1010 & 32.6936 & -6.5926 & \textbf{-40.1225} \\ \midrule[0.5pt]
	
		\parbox{4cm}{HF (STO-3G, 9)} 
		& -79.3617 &  * & * & 13.4477 & 26.1872 & * & * & -39.7267
		\\ \midrule[0.5pt]
	
		\parbox{4cm}{HF (6-31G, 17)} 
		& -79.6901 &  * & * & 13.4477 & 26.0619 & * & * & \textbf{-40.1804}
		\\ \midrule[0.5pt]
	
		\parbox{4cm}{HF (6-311G, 25)} 
		& -79.6872 &  * & * & 13.4477 & 26.0515 & * & * & -40.1880
		\\ \midrule[0.5pt]
	
		% \parbox{3cm}{d4ft} 
		% & -79.503407 & * & * & 13.349528 & * & 32.452280 & -5.856054 & -39.467655
		% \\ \midrule[0.5pt]
	
		% \parbox{3cm}{pyscf} 
		% & -79.472583 & * & * & 13.349528 & * & 32.586078 & -5.843392 & -39.290969
		% \\ \midrule[0.5pt] \midrule[0.5pt]
	
		% \parbox{3cm}{OUR (\ce{H2O})} & -120.2329 & 68.1350 & -188.3679 & 9.1895 &
		% 37.4786 & 46.2207 & -8.7421 & -73.5648 \\ \midrule[0.5pt]
	
		% \parbox{3cm}{$n_g = 10, \epsilon=10^{-3}$} & -120.9352 & 74.8185 &
		% -195.7537 & 9.1895 & 37.0497 & 46.0144 &
		% -8.9647 & -74.6960 \\ \midrule[0.5pt]
	
		% \parbox{3cm}{$n_g = 20, \epsilon=10^{-3}$} & -122.6473 & 75.5681 &
		% -198.2154 & 9.1895 & 37.7575 & 46.6995 &
		% -8.9420 & -75.7003 \\ \midrule[0.5pt]
	
		% \parbox{3cm}{$n_g = 20, \epsilon=10^{-4}$} & -122.7171 & 75.8241 &
		% -198.5412 & 9.1895 & 37.8074 & 46.7600 &
		% -8.9526 & -75.7202 \\ \midrule[0.5pt]
	
		% \parbox{3cm}{$n_g = 10, \epsilon=10^{-4}$} & -119.5637 & 74.4819 &
		% -194.0456 & 9.1895 & 36.1289 & 45.0722 &
		% -8.9433 & -74.2454 \\ \midrule[0.5pt]
	
		\parbox{4cm}{OUR ($\ce{H2O}$), 22} & -122.9802 & 75.9462 &
		-199.0371 & 9.1895 & 37.8786 & 46.8579 &
		-8.9601 & \textbf{-76.0036} \\ \midrule[0.5pt]
	
		\parbox{4cm}{HF (STO-3G, 7)} 
		& -122.3614 &  * & * & 9.1895 & 38.2089 & * & * & -74.9630
		\\ \midrule[0.5pt]
	
		\parbox{4cm}{HF (6-31G, 13)} 
		& -122.9701 &  * & * & 9.1895 & 37.7966 & * & * & -75.9839
		\\ \midrule[0.5pt]
	
		\parbox{4cm}{HF (6-311G, 19)} 
		& -123.0146 &  * & * & 9.1895 & 37.8157 & * & * & \textbf{-76.0094}
		\\ \midrule[0.5pt]
	
		% \parbox{3cm}{d4ft} 
		% & -122.983359 & * & * & * & 9.189533 & 46.449287 & -8.117571 & -75.462110
		% \\ \midrule[0.5pt]
	
		% \parbox{3cm}{pyscf} 
		% & -122.839414 & * & * & * & 9.189534 & 46.589372 & -8.093291 & -75.153800
		% \\ \midrule[0.5pt]
	
		\\\bottomrule[1.5pt]
		\end{tabular}
		}
	\end{table}
\end{frame}


% \begin{frame}{Next step}
% 	\begin{itemize}
% 		\item 1. Compare the scalability of the double-electron integral
% 		calculation over the number of basis functions with the SOTA method, e.g.
% 		MD, HGP(OS), Rys.
% 		\item 2. Conduct numerical experiments to check if the adaptive
% 		Gaussian basis can achieve the same accuracy with a smaller number of
% 		basis.
% 		\item 3. Implement post HF method, e.g. CCSD, CCSD(T).
% 	\end{itemize}
% \end{frame}


\begin{frame}{A primer on plane-wave based solid-state DFT}
	\begin{itemize}
		\item Wavefunctions, density, and potential functions are parametrized as the
		linear combination of plane-waves $e^{i\mathbf{k} \cdot \mathbf{r}}, \mathbf{k}$
		lattice vector.
		\item Fast Fourier transform enables efficient calculation of different
		parts of the energy.
		\item Systematical approach to obtain more accurate DFT calculation -- increase
		the plane-wave energy cutoff.
	\end{itemize}
	
	\begin{itemize}
		\item Energy cutoff has to be very high around the atomic site, leading to unaffordable
		computational budget for the whole supercell.
	\end{itemize}
	% \JX{We need to survey the literature for some multiscale basis such as
	% the wavelet basis.}
\end{frame}


\begin{frame}{Literature}
	\begin{itemize}
		\item Use fixed analytic transformation mapping depending on the position of the
		atom. (Similar to the r type mesh refinement in numerical PDE.)
		\item Solve some OT-type problems to predetermine the basis before the SCF
		loop.
	\end{itemize}
	
	The design of the distorted grid has great freedom and has to be done by expert 
	for each crystal system -- can we have a more adaptive and automatic design
	of the distorted grid?
\end{frame}


\begin{frame}{Normalizing flow on periodic domain}

	\begin{itemize}
		\item Invertible mapping on $S^1$ parametrized by spline function
		\begin{figure}[h]
			\centering
			\includegraphics[width=.4\linewidth]{fig/nsf.jpg}
		\end{figure}
		\item Coupling layer to handle 3D domain:
		\begin{equation*}\label{equ:cnf-auto-regressive}
			\begin{pmatrix}
			x_1^{(0)} \\
			x_2^{(0)}
			\end{pmatrix} 
			\underset{x_2^{(1)}=\ x_2^{(0)}}{\xrightarrow{x_1^{(1)}=\ \phi(x_1^{(0)}, \text{NN}^{(0)}_1(t))}}
			\begin{pmatrix}
			x_1^{(1)} \\
			x_2^{(1)}
			\end{pmatrix} 
			\underset{x_2^{(2)}=\ \phi(x_2^{(1)},\, \text{NN}^{(1)}_2(x_1^{(1)},t))}{\xrightarrow{\quad\quad x_1^{(2)} =\ x_1^{(1)} \quad\quad }}
			\begin{pmatrix}
			x_1^{(2)} \\
			x_2^{(2)}
			\end{pmatrix}.
		\end{equation*}
		\item Autoregressive layer to improve expressivity.
	\end{itemize}
\end{frame}


\begin{frame}{Distorted plane-wave basis}
	\begin{itemize}
		\item Distorted planewaves (DPW) $\phi _{\vb{G}}$ is defined as planewaves in the parameter space
		\begin{equation*} \label{eqn:dpw}
		\braket{\vb{r}}{\vb{G}} := \phi _{\vb{G}}(\vb{r})
			= \frac{1}{\sqrt{\Omega_{\vb*{\xi }} }} \abs{ J_{f^{-1}}(\vb{r})}^{\frac{1}{2}} \exp \left[\I \vb{G} ^{\top} f^{-1}(\vb{r})  \right]
		\end{equation*}
		\item Ideally, the grid points will concentrate around the atomic sites and be
		relatively sparse in the interatomic region.
	\end{itemize}
\end{frame}


\begin{frame}{Kinetic matrix element: FFT}
	\begin{itemize}
		\item \begin{equation*}
			\begin{split}
		& \mel{\psi _{i, \vb{k}}}{\hat{T}}{\psi _{j, \vb{k}}} \\
				=& \frac{1}{2} \sum_{\vb{G}', \vb{G}} c^{*}_{i, \vb{k}, \vb{G}'} c_{j, \vb{k}, \vb{G}}\{B_{\vb{G}', \vb{G}} - G_{q}[U_{\vb{G}', \vb{G}}]_{q} + G'_{q}[U_{\vb{G}', \vb{G}}]_{q} + (G_{p}+k_{p})(G'_{q}+k_{q})[P_{\vb{G}', \vb{G}}]_{pq}\} .
			\end{split}
		\end{equation*}
		\item FFT implementation of the Toeplitz matrix-vector product is extremely fast.
	\end{itemize}
\end{frame}


\begin{frame}{Hartree energy: solving the Poisson equation}
	\begin{equation*}
		\laplacian V_{\rho _{1}} = -4\pi \rho_{1}.
	\end{equation*}
	\begin{itemize}
		\item Solved by FFT in plane-wave basis;
		\item Solved by physics-informed neural network -- introducing an inner
		loop of optimization
	\end{itemize}
\end{frame}


\begin{frame}{Overall algorithm}
	\begin{algorithm}[H]
		\caption{Distorted plane-wave method}
		\begin{algorithmic}[1]
			\State \textbf{Input:} 
		\end{algorithmic}
	\end{algorithm}
\end{frame}


\begin{frame}{Good features}
	There are several advantages of using the distorted plane-wave method.
	\begin{itemize}
		\item The orthonormality of the basis is preserved.
		\item Calculation of the density related energy can be estimated 
		efficiently by the Monte-Carlo method instead of integrating over the
		grid:
		\begin{equation*}
			E_{\text{xc}} = \int_{\Omega} \epsilon_{\text{xc}}^{\text{LDA}}(\rho(\mathbf{r}))
			\rho(\mathbf{r})d\mathbf{r} = \frac{1}{N} \sum_{i=1}^N \epsilon_{\text{xc}}^{\text{LDA}}(\rho_i).
		\end{equation*}
	\end{itemize}
\end{frame}


\begin{frame}{Numerical results}
	\begin{table}[htbp]
		\footnotesize
	\centering
	\caption{Band structure calculation for diamond with LDA. All energies are in eV unit.}
	{\tabulinesep=1.2mm
	\begin{tabu}{cccccc}
	\hline
	Method & FFT grid size & band gap & L & X & $\Gamma$  \\ \hline\hline
	\multirow{3}{*}{PW}
	 & 128 & 3.05869 & 7.51526 & 3.05869 & 4.79634  \\ \cline{2-6}
	 & 96  & 3.14795 & 7.61833 & 3.14795 & 4.77946  \\ \cline{2-6}
	 & 64  & 3.68519 & 7.72595 & 3.68519 & 4.88759  \\ \cline{2-6}
	\hline\hline
	\multirow{3}{*}{PW + ANC}
	 & 96 & 3.08192 & 7.7782  & 3.08192 & 4.75455  \\ \cline{2-6}
	 & 64 & 3.10424 & 7.77621 & 3.10424 & 4.88761  \\ \cline{2-6}
	 & 48 & 1.98104 & 7.39363 & 1.98104 & 4.84532  \\ \cline{2-6}
	\hline\hline
	\multirow{1}{*}{FDPW + PINN}
	 & 48 & 3.01882 & 7.26194 & 3.01882 & 4.41584  \\ \cline{2-6}
	\hline
	\end{tabu}}
	\end{table}
\end{frame}


\begin{frame}{Pseudopotential: domain decomposition}
	\begin{itemize}
		\item Near atomic sites, influence of other atoms is negligible -- 
		atomic configuration;
		\item At the interatomic region, the wavefunction is smooth and plane-wave
		is efficient;
		\item Need a clever way to glue together above two parts of the wavefunctions.
	\end{itemize}
\end{frame}


\begin{frame}{Diagrammtic illustration of PAW}
	Projector augmented-wave method (PAW)

	All the quantites: wavefunctions, density, energy can be calculated via the
	following scheme:
	\begin{figure}[h]
		\centering
		\includegraphics[width=.4\linewidth]{fig/paw.jpg}
	\end{figure}
\end{frame}


\begin{frame}{Mathematical essence of PAW}
	\begin{equation*}
			\begin{aligned}
				& \text{Wavefunctions:} \qquad \psi & = \wtd \psi + \sum_a (\psi^a - \wtd\psi^a),		\\
				& \text{Density:} \qquad \rho = & \ \wtd \rho + \sum_a (\rho^a - \wtd \rho^a),		\\
				& \text{Energy:} \qquad E_{\text{kin}} = & \ \wtd E_{\text{kin}} +
				\half\sum_{a,i,j}D_{ij}^a
				\int_{\mathbb{S}^a}\lp \nabla{\phi_i^{a}}^*(\mathbf{r}) \nabla\phi_j^a(\mathbf{r}) - 
				\nabla\wtd {\phi_i^{a}}^*(\mathbf{r}) \nabla\wtd\phi_j^a(\mathbf{r}) \rp d\mathbf{r},		\\
			\end{aligned}
	\end{equation*}
\end{frame}


\begin{frame}{MLXC under the PAW formulation}
	\textbf{\color{red} Why do we need the PAW's formulation?}
	
\end{frame}

\begin{frame}{Difficulties in the PAW implementation}
	\begin{itemize}
		\item Mathematical sophistication;
		\item Misalignment between the pseudopotential file and the PAW formulation;
		\item Design of the unit in the code development of the PAW
	\end{itemize}
\end{frame}


\begin{frame}
	Thank you for your attention!
	Q \& A
\end{frame}

\begin{frame}{Many-body Schr\"odinger equation*}
	After Born–Oppenheimer approximation, the many-body Schr\"odinger equation
	is given by:
	\begin{equation*}
		\begin{aligned}
			\hat{\opH^e} & \Psi(\mfr_1, \mfr_2, ..., \mfr_N) = E
			\Psi(\mfr_1, \mfr_2, ..., \mfr_N),		\\
			\hat{\opH^e} & = -\half \Delta_{\mfr} - \sum_{i=1}^N\sum_{j=1}^{n_A} \frac{Z_j}
			{\norml \mfr_i - \mfR_j \normr_2} + \frac{1}{2} \sum_{i \neq j}^N
			\frac{1}{r_{ij}}		\\
			& = \sum_{i=1}^N \wht{\operatorname{h}}(\text{i}) + \frac{1}{2} \sum_{i \neq j}^N
			\frac{1}{r_{ij}}
		\end{aligned}
	\end{equation*}
	Solving this equation, we can obtain the ground state energy, potential
	energy surface (for geometry optimization and transition states search),
	band structure for the solid state system, and etc.
\end{frame}

% for long talk
\begin{frame}{Mean-field approximation: Slater determinant}
	Pauli exclusion principle requires following antisymmetric constraint on the
	many-body wave function:
	\begin{equation*}
		\Psi(..., \mfr_j, ..., \mfr_i, ...) = 
		- \Psi(..., \mfr_i, ..., \mfr_j, ...)
	\end{equation*}

	The many-body wave function of the Hartree-Fock theory is given by a single
	Slater determinant:
	\begin{equation*}
		\Psi_{\text{HF}}(\mfr_1, ..., \mfr_N) = \frac{1}{\sqrt{N!}} 
		\begin{vmatrix}
			\phi_1(\mfr_1) & \phi_1(\mfr_2) & \cdots & \phi_1(\mfr_N) \\
			\phi_2(\mfr_1) & \phi_2(\mfr_2) & \cdots & \phi_2(\mfr_N) \\
			\vdots & \vdots & \ddots & \vdots \\
			\phi_N(\mfr_1) & \phi_N(\mfr_2) & \cdots & \phi_N(\mfr_N)
		\end{vmatrix}
	\end{equation*}
\end{frame}

% for long talk
\begin{frame}{Total energy}
	\begin{equation*}
		\begin{aligned} & E[\Psi^{\text{HF}}] = 
			\left\langle\Psi^{\text{HF}}|\hat{\opH^e}|\Psi^{\text{HF}}
			\right\rangle =  \sum_{i=1}^N \int\text{d}\mathbf{r}_i
		  \phi_i^*(\mathbf{r}_i)
			\wht{\operatorname{h}}(\text{i}) \phi_i(\mathbf{r}_i) \\ 
			&+ \frac{1}{2N(N - 1)} \sum_{i \neq j}^N \sum_{k \neq l}^N \iint
			\mathrm{d}\mathbf{r}_i
			\text{d}\mathbf{r}_j \phi_k^*(\mathbf{r}_i)\phi_l^*(\mathbf{r}_j)
			\frac{1}{|\mathbf{r}_i-\mathbf{r}_j|}\phi_k(\mathbf{r}_i)
			\phi_l(\mathbf{r}_j) \\ 
			&- \frac{1}{2N(N - 1)} \sum_{i \neq j}^N \sum_{k \neq l}^N \iint
			\text{d}\mathbf{r}_i
			\text{d}\mathbf{r}_j\phi_k^*(\mathbf{r}_i)\phi_l^*(\mathbf{r}_j)
			\frac{1}{|\mathbf{r}_i-\mathbf{r}_j|}\phi_k(\mathbf{r}_j)
			\phi_l(\mathbf{r}_i)  \\
			& = E(\phi_1, \phi_2, \cdots, \phi_N).
		\end{aligned}
	\end{equation*}
	Minimizing the total energy will give the ground state configuration which is
	consistent with the self-consistent field approach. The double-electron
	integrations $O(N^4)$ are the dominant part of computational cost.
\end{frame}

\begin{frame}{SCF}
	Self-consistent field approach to HF: adding Lagrangian multiplier
	(molecular orbital energy) and the KKT condition translates to an eigenvalue
	problem:
	\begin{equation*}
		\begin{aligned}
			\delta E[\phi_k^*(x_k)] & \ = \delta E(\phi_1, \phi_2, \cdots, \phi_N)
			- \delta\left[\sum_{i=1}^N \sum_{j=1}^N
			\lambda_{ij} \left( \left\langle\phi_i, \phi_j\right\rangle - \delta_{ij}
			\right)\right]		\\
			\wht \opF(\text{k})\phi_k(\mathbf{r}_k) & \equiv \left[ \hat{h}
			(\text{k}) + \hat{J}(\text{k}) - \hat{K}(\text{k}) \right]
			\phi_k(\mathbf{r}_k) = \epsilon_k \phi_k(\mathbf{r}_k),	\\
			\wht \opJ(\text{k}) & \equiv \sum_{j=1}^N \int \mathrm{d}\mathbf{r}_j
			\frac{\phi_j^*(\mathbf{r}_j) \phi_j(\mathbf{r}_j)}{|\mathbf{r}_k-
			\mathbf{r}_j|}= \sum_{j=1}^N \int \mathrm{d}\mathbf{r}_j \frac{\rho(
			\mathbf{r}_j)}{|\mathbf{r}_k-\mathbf{r}_j|},\\
			\wht \opK(\text{k})\phi_{k}(\mathbf{r}_k) & \equiv  \sum_{j=1}^N
			\phi_{j}(\mathbf{r}_k) \int \text{d}\mathbf{r}_j \frac{\phi_j^*
			(\mathbf{r}_j) \phi_k(\mathbf{r}_j)}{|\mathbf{r}_k-\mathbf{r}_j|}.\\
		\end{aligned}
	\end{equation*}
\end{frame}

\begin{frame}{Potential improvement}
	\begin{itemize}
		\item 1. Using greedy algorithm to optimize for more Gaussian modes, i.e.
		one can use the optimization results for less modes as the warm-start
		for more modes.
		\item 2. The approximation interval $[0, M]$ can be done adaptively according
		to the relative position of two Gaussian modes.
		\item 3. Motivated by the Boys function, can we design other decomposition
		of the Coulomb kernel?
		\begin{equation*}
			\frac{1}{r} = C\int_{\mbR} e^{-r^2t^2} dt.
		\end{equation*}
	\end{itemize}
\end{frame}


% \begin{frame} % Use [allowframebreaks] to allow automatic splitting across slides if the content is too long
%     \frametitle{References}
 
%     \begin{thebibliography}{99} % Beamer does not support BibTeX so references must be inserted manually as below, you may need to use multiple columns and/or reduce the font size further if you have many references
%         \footnotesize % Reduce the font size in the bibliography
 
% 		\bibitem[migbs]{migbs}
% 		Gill, Peter MW.
% 		\newblock Molecular integrals over gaussian basis functions.
% 		\newblock \emph{Advances in quantum chemistry. Vol. 25. Academic Press,
% 		1994. 141-205.}

% 		\bibitem[3dgs]{3dgs}
% 		Kerbl, Bernhard, et al
% 		\newblock 3D Gaussian Splatting for Real-Time Radiance Field Rendering
% 		\newblock \emph{ACM Trans. Graph. 42.4 (2023): 139-1.}

% 		\bibitem[S.A. and Q.L., 2024]{nigbms}
% 		S. Arisaka and Q. Li (2024)
% 		\newblock Accelerating Legacy Numerical Solvers by Non-intrusive Gradient-based Meta-solving
% 		\newblock \emph{International Conference on Machine Learning 2024}
 
        
%     \end{thebibliography}
% \end{frame}

\end{document}