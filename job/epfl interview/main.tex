\documentclass[paper slide]{beamer}
\usetheme{Boadilla}
\usepackage{essay-def}
\usepackage{bm}
\usepackage{amsfonts}
\usepackage{amssymb}
\usepackage{amsmath}
\usepackage{amsthm}
\usepackage{comment}
\usepackage{subcaption}
\usepackage{geometry}
\usepackage{algorithmic}
\usepackage{algpseudocode}
\usepackage{algorithmicx}
\geometry{left=1cm,right=1cm}
    \title[Interview]{Interview for MatMat group}
\author[J. Zhao]{Jiaxi Zhao}
\date[\today]{}
\begin{document}
\par \setlength{\parindent}{2em}

\begin{frame}
\titlepage
\end{frame}


\begin{frame}{}
	\noindent
	\small
	Many machine-learning assisted scientific computing methods have an
	iterative nature, e.g., turbulence modeling, XC functional:
  \begin{equation*}
         \partial_t\mfu = \mcL(\mfu, \mfy, t), \quad \mfy = \phi(\mfu, t),
				 \quad \min_{\theta} \sum_{n} \norml \phi_{\theta}(\mfu^{(n)}) - \mfy^{(n)} \normr^2,
	\end{equation*}
	The stability and a-posteriori performance are not satisfactory.
	\begin{itemize}
		\item 1. Tangent-space regularization method (\textit{published in SISC})
		\begin{itemize}
			\item · First mathematical formulation and analysis.
			\item · Non-intrusive and differentiable regularization.
			\item · Significant improvement over dynamics-agnostic methods.
			\item · Deployment to practical urban environment simulation (ongoing).
		\end{itemize}
		\item 2. Generative subgrid-scale model (\textit{accepted by ICLR 2025 MLMP})
		\item 3. Numerical anlysis of the turbulence modeling (multiscale,
		data-imbalance, multivaluedness)
		\item 4. Distorted plane-wave via normalizing flow (adaptive basis set, smaller
		cutoff energy, comparable accuracy)
		\item 5. Adaptive Gaussian basis set (efficient electron integral, differentiable basis sets)
	\end{itemize}
\end{frame}
 
 
\begin{frame}{Mathematics, physics, and programming background}
	\noindent
  Mathematics \& Physics:
	\begin{itemize}
    \item 1. Mathematics, graduate level course on computational math and analysis: 
		numerical analysis, (numerical) PDE, numerical linear algebra, stochastic analysis, etc.
		\item 2. Physics (quantum mechanics, quantum chemistry, solid-state physics, fluid dynamics).
  \end{itemize}
	\noindent
	Programming:
	\begin{itemize}
    \item 1. Core contributor of the \textit{Jrystal} package (pseudopotential,
		accuracy test modules, exploring the machine-learning XC functionals).
    \item 2. Extensive experience with deep-learning framework and models (PyTorch, JAX, generative models,
		differentiable programming).
    \item 3. Familiar with various open-source packages (PySCF, OpenFOAM).
  \end{itemize}
\end{frame}

 
\begin{frame}{Research vision and contribution to MatMat group}
	\noindent
	Research vision:
	\begin{itemize}
		\item Machine-learning tools can advance scientific computing (differentiable programming,
		data-centric viewpoint, generative modeling)
		\item Numerical analysis for hybrid algorithms (stability analysis, error estimation, 
		uncertainty quantification)
	\end{itemize}
	I aim to contribute to the MatMat group research in the following two aspects:
	
	1. Gradient-accelerated inverse materials design 
		\begin{itemize}
			\item · Combining stabilization algorithm with inverse design.
			\item · Generative model guided by differentiable DFT calculation.
		\end{itemize}
	
	2. Estimation of simulation errors
		\begin{itemize}
			\item · Numerical analysis related to pseudo-potential and XC functional.
			\item · Machine-learning XC functionals and its related problems.
		\end{itemize}
\end{frame}


% \begin{frame} % Use [allowframebreaks] to allow automatic splitting across slides if the content is too long
%     \frametitle{References}
 
%     \begin{thebibliography}{99} % Beamer does not support BibTeX so references must be inserted manually as below, you may need to use multiple columns and/or reduce the font size further if you have many references
%         \footnotesize % Reduce the font size in the bibliography
 
% 		\bibitem[BDI]{bdi}
% 		M. Benjamin, S. Domino, and G. Iaccarino
% 		\newblock Neural Networks for Large Eddy Simulations of Wall-bounded Turbulence: Numerical Experiments and Challenges
% 		\newblock \emph{The European Physical Journal E}

% 		\bibitem[Zhao 2024]{ds}
%         J. Zhao and Q. Li (2024)
%         \newblock Mitigating Distribution Shift in Machine Learning-augmented Hybrid Simulation
%         \newblock \emph{Arxiv preprint https://arxiv.org/pdf/2401.09259}

%         \bibitem[S.A. and Q.L., 2024]{nigbms}
%         S. Arisaka and Q. Li (2024)
%         \newblock Accelerating Legacy Numerical Solvers by Non-intrusive Gradient-based Meta-solving
%         \newblock \emph{International Conference on Machine Learning 2024}
 
        
%     \end{thebibliography}
% \end{frame}

\end{document}