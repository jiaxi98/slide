\documentclass[paper slide]{beamer}
\usetheme{Boadilla}
\usepackage{essay-def}
\usepackage{bm}
\usepackage{amsfonts}
\usepackage{amssymb}
\usepackage{amsmath}
\usepackage{amsthm}
\usepackage{comment}
\usepackage{subcaption}
\usepackage{geometry}
\usepackage{algorithmic}
\usepackage{algpseudocode}
\usepackage{algorithmicx}
\geometry{left=1cm,right=1cm}
    \title[Probabilistic SGS modeling]{A probabilistic approach to 
		subgrid-scale modeling}
\author[J. Zhao]{Jiaxi Zhao \\ \small joint with Q. Li\@ NUS, N. Thuerey\@ TUM}
\date[\today]{NUS-SJTU PhD Forum \\ \today}
\begin{document}
\par \setlength{\parindent}{2em}

\begin{frame}
\titlepage
\end{frame}

\begin{frame}{PhD project 1}
	\begin{enumerate}
		\item Establish a theoretical understanding to explain the mismatch bewteen the a-priori
		and a-posteriori error appears in scientific machine learning.
		\item Propose the tangent-space regularization method to stabilize the 
		\item Propose a generative modeling approach to learn the subgrid-scale (SGS) model
		and use this methods to improve the accuracy of long time simulation of the chaotics systems.
	\end{enumerate}
	\footnotetext{}
\end{frame}


\begin{frame}{PhD project 2}
	\begin{enumerate}
		\item 
		\item Propose a generative modeling approach to learn the SGS model
	\end{enumerate}
\end{frame}


\begin{frame}{Mathematics, Physics, and coding}
	\begin{enumerate}
		\item Strong background in mathematics, with abundant experience in numerical methods,
		especially in density functional theory and computational fluid dynamics.
		\item Strong skills in differentiable programming.
	\end{enumerate}
\end{frame}

\begin{frame}{My insight in MatMat}
	\begin{enumerate}
		\item The DFT and material design pipeline is also dynamics, similar to the problems
		framework I have been working on during my graduate study.
		\item I hope that my solid mathematics background can help to explore the problems we
		encountered during the material design results.
	\end{enumerate}
\end{frame}



% \begin{frame} % Use [allowframebreaks] to allow automatic splitting across slides if the content is too long
%     \frametitle{References}
 
%     \begin{thebibliography}{99} % Beamer does not support BibTeX so references must be inserted manually as below, you may need to use multiple columns and/or reduce the font size further if you have many references
%         \footnotesize % Reduce the font size in the bibliography
 
% 		\bibitem[BDI]{bdi}
% 		M. Benjamin, S. Domino, and G. Iaccarino
% 		\newblock Neural Networks for Large Eddy Simulations of Wall-bounded Turbulence: Numerical Experiments and Challenges
% 		\newblock \emph{The European Physical Journal E}

% 		\bibitem[Zhao 2024]{ds}
%         J. Zhao and Q. Li (2024)
%         \newblock Mitigating Distribution Shift in Machine Learning-augmented Hybrid Simulation
%         \newblock \emph{Arxiv preprint https://arxiv.org/pdf/2401.09259}

%         \bibitem[S.A. and Q.L., 2024]{nigbms}
%         S. Arisaka and Q. Li (2024)
%         \newblock Accelerating Legacy Numerical Solvers by Non-intrusive Gradient-based Meta-solving
%         \newblock \emph{International Conference on Machine Learning 2024}
 
        
%     \end{thebibliography}
% \end{frame}

\end{document}