\documentclass[paper slide]{beamer}
\usetheme{Boadilla}
\usepackage{essay-def}
\usepackage{bm}
\usepackage{amsfonts}
\usepackage{amssymb}
\usepackage{amsmath}
\usepackage{amsthm}
\usepackage{comment}
\usepackage{subcaption}
\usepackage{geometry}
\usepackage{algorithmic}
\usepackage{algpseudocode}
\usepackage{algorithmicx}
\geometry{left=1cm,right=1cm}
    \title[Data-driven SGS modeling]{Data-driven subgrid-scale modeling for wall-bounded turbulence}
\author[J. Zhao]{Jiaxi Zhao \\ \small joint with S. Arisaka, Q. Li, T. Hasama, N. Ikegaya, W. Wang \\
\small NUS \& Kajima \& KU}
\date[\today]{The 19th OpenFOAM Workshop \\ \today}
\begin{document}
\par \setlength{\parindent}{2em}

\begin{frame}
\titlepage
% \begin{figure}[ht]
% 	\centering
% 	\begin{subfigure}[b]{0.32\textwidth}
% 		\includegraphics[width=.8\textwidth]{fig/nus-logo-horizon.png}
% 	\end{subfigure}
% 	\begin{subfigure}[b]{0.32\textwidth}
% 		\includegraphics[width=.8\textwidth]{fig/kajima-logo.jpg}
% 	\end{subfigure}
% 	\begin{subfigure}[b]{0.32\textwidth}
% 		\includegraphics[width=.8\textwidth]{fig/ku_logo.jpg}
% 	\end{subfigure}
% \end{figure}
\end{frame}

\begin{frame}{Motivation}
	The effect cutoff used by the plane-wave method to solve large system is
	unaffordable. Moreover, given the nature of the electron density which will
	be concentrated around the nucleus, while away from the nucleus which is the
	case of majority of the computational region, the electron density is
	relatively smooth, the plane-wave method is not efficient.
	\JX{We need to survey the literature for some multiscale basis such as
	the wavelet basis.}

\end{frame}

\begin{frame}{Literature}
	In literature, there has been lots of work on exploring the adaptive basis for
	solving DFT. 
	\begin{itemize}
		\item Use fixed analytic transformation mapping depending on the position of the
		atom. (Similar to the r type mesh refinement in numerical PDE.)
		\item Solve some OT-type problems to predetermine the basis before the SCF
		loop.
	\end{itemize}
	However, our method is different from the above method in several aspects:
	\begin{itemize}
		\item Our basis is continuously optimized during the optimization porcess of
		DFT.
		\item We use Monte-Carlo method to estimate the density related energy,
		differs from the traditional PDE solving method.
	\end{itemize}
\end{frame}

\begin{frame}{Advantages}
	There are several advantages of using the distorted plane-wave method.
	\begin{itemize}
		\item The orthognormality of the basis is preserved.
		\item Calculation of the density related energy can be estimated 
		efficiently by the Monte-Carlo method instead of solving the PDE.
	\end{itemize}
\end{frame}

\begin{frame} % Use [allowframebreaks] to allow automatic splitting across slides if the content is too long
    \frametitle{References}
 
    \begin{thebibliography}{99} % Beamer does not support BibTeX so references must be inserted manually as below, you may need to use multiple columns and/or reduce the font size further if you have many references
        \footnotesize % Reduce the font size in the bibliography
 
		\bibitem[BDI]{bdi}
		M. Benjamin, S. Domino, and G. Iaccarino
		\newblock Neural Networks for Large Eddy Simulations of Wall-bounded Turbulence: Numerical Experiments and Challenges
		\newblock \emph{The European Physical Journal E}

		\bibitem[Zhao 2024]{ds}
        J. Zhao and Q. Li (2024)
        \newblock Mitigating Distribution Shift in Machine Learning-augmented Hybrid Simulation
        \newblock \emph{Arxiv preprint https://arxiv.org/pdf/2401.09259}

        \bibitem[S.A. and Q.L., 2024]{nigbms}
        S. Arisaka and Q. Li (2024)
        \newblock Accelerating Legacy Numerical Solvers by Non-intrusive Gradient-based Meta-solving
        \newblock \emph{International Conference on Machine Learning 2024}
 
        
    \end{thebibliography}
\end{frame}

\end{document}